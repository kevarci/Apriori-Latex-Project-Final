\documentclass{beamer}

% --- Theme and Color Settings ---
\usetheme{Frankfurt}
\usecolortheme{whale}
\setbeamertemplate{navigation symbols}{}

% --- Required Packages ---
\usepackage[utf8]{inputenc}
\usepackage[T1]{fontenc}
\usepackage[spanish]{babel}
\usepackage{graphicx}
\usepackage{amsmath, amsfonts, amssymb}
\usepackage{tikz}
\usetikzlibrary{shapes.geometric, arrows, positioning}

% --- Title Information ---
\title{Potenciando el Marketing con Apriori}
\subtitle{Transformando datos en estrategias efectivas}
\author{
    \begin{tabular}{c} 
    Fernanda Flores \\ 
    Kevin Arciniegas \\ 
    Giussepe Marrero
    \end{tabular}
}
\institute{4Geeks Academy}
\date{\today}

% --- Document Start ---
\begin{document}

% --- Slide 1: Title Slide ---
\begin{frame}
    \titlepage
\end{frame}

% --- Slide 2: Table of Contents ---
\begin{frame}{Contenido}
    \tableofcontents
\end{frame}

% --- Slide 3: Impactful Introduction ---
\section{Introducción}
\begin{frame}{¿Sabías esto?}
    \centering
    \Large
    \textbf{El 80\% de las decisiones de compra están influenciadas por recomendaciones personalizadas.}
\end{frame}

% --- Slide 4: Project Purpose ---
% Agregar al principio del documento, después de los otros paquetes
\usepackage{fontspec}
\usepackage{emoji}

% O alternativamente, reemplazar el emoji por texto
% Modificar estas diapositivas:
\begin{frame}{¿Qué queremos lograr?}
    \centering
    \Large
    Este proyecto explora cómo convertir datos transaccionales en estrategias efectivas.  
    \vspace{1em}
    Te explicamos cómo. $\rightarrow$
\end{frame}

% --- Slide 5: Contextualizing the Problem ---
\section{Market Basket Analysis}
\begin{frame}{Market Basket Analysis: ¿Por qué es relevante?}
    \textbf{Definición:}  
    Técnica clave en retail y comercio electrónico para descubrir relaciones entre productos comprados juntos.  
    \begin{itemize}
        \item ¿Qué patrones de compra están escondidos en los datos?
        \item Ejemplo: "Si un cliente compra pan y mantequilla, ¿también comprará leche?"
        \item Aplicación: Optimizar ventas cruzadas, e-commerce y estrategias de marketing.
    \end{itemize}
\end{frame}

\section{Algoritmo Apriori}
\subsection{Conceptos Básicos}
\begin{frame}{¿Qué es Apriori?}
    \textbf{Una solución basada en patrones:}
    \begin{itemize}
        \item Escanea transacciones para identificar productos populares.
        \item Construye combinaciones (pares, tríos, etc.).
        \item Descubre reglas como: "Si compras X, es probable que lleves Y."
    \end{itemize}
    \textbf{Un ejemplo real:}
    "Café y galletas son frecuentemente comprados juntos, generando una regla de recomendación clave."
\end{frame}

\subsection{Funcionamiento}
\begin{frame}{¿Cómo Funciona el Algoritmo?}
    \textbf{Pasos Clave de Apriori:}  
    \begin{itemize}
        \item \textbf{Escaneo de transacciones:} Identifica ítems frecuentes de tamaño 1.
        \item \textbf{Generación de combinaciones:} Forma pares/tríos candidatos aplicando reglas de poda.
        \item \textbf{Evaluación de métricas:} Calcula soporte, confianza y lift.
        \item \textbf{Generación de reglas:} Deriva asociaciones accionables.
    \end{itemize}
\end{frame}

\begin{frame}{Diagrama de Flujo}
    \centering
    \begin{tikzpicture}[node distance=1.2cm]
        \tikzstyle{startstop} = [rectangle, rounded corners, minimum width=2.8cm, minimum height=0.7cm, text centered, draw=black, fill=orange!30]
        \tikzstyle{process} = [rectangle, minimum width=2.8cm, minimum height=0.7cm, text centered, draw=black, fill=cyan!20]
        \tikzstyle{arrow} = [thick,->,>=stealth]

        \node (start) [startstop] {Dataset Inicial};
        \node[process, below=1cm of start] (prep) {Itemsets Frecuentes};
        \node[process, below=1cm of prep] (matrix) {Generar Candidatos};
        \node[process, below=1cm of matrix] (evaluate) {Evaluar Métricas};
        \node[startstop, below=1cm of evaluate] (rules) {Generar Reglas};

        \draw[arrow] (start) -- (prep);
        \draw[arrow] (prep) -- (matrix);
        \draw[arrow] (matrix) -- (evaluate);
        \draw[arrow] (evaluate) -- (rules);
    \end{tikzpicture}
\end{frame}

\section{Implementación}
\begin{frame}{De los Datos a la Acción}
    \textbf{Cómo resolvemos el problema:}  
    Nuestra implementación híbrida (Python + R):  
    \begin{itemize}
        \item Limpieza de datos masivos en Python.
        \item Algoritmo Apriori en R con precisión estadística.
        \item Dashboards interactivos para visualizar asociaciones clave.
    \end{itemize}
\end{frame}

\section{Resultados}
\begin{frame}{Ejemplos Visuales}
    \textbf{Gráficas interactivas:}  
    \begin{itemize}
        \item Scatter plot que muestra lift y confianza.
        \item Ejemplo: "72\% de los clientes que compran café, también compran galletas."
    \end{itemize}
\end{frame}

\section{Conclusión}
\begin{frame}{Nuestra Promesa: Decisiones Basadas en Datos}
    \centering
    \Huge
    \textbf{De transacciones a estrategias.}
    
    \vspace{1em}
    \Large
    ¿Estás listo para transformar tu negocio con ciencia de datos? Hablemos. $\rightarrow$
\end{frame}

\end{document}
