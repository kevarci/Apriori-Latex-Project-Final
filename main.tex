\documentclass{beamer}
% --- Required Packages ---
\usepackage[utf8]{inputenc}
\usepackage[T1]{fontenc}
\usepackage[spanish]{babel}
\usepackage{graphicx}
\usepackage{amsmath, amsfonts, amssymb}
\usepackage{tikz}
\usetikzlibrary{shapes,shapes.geometric, arrows.meta, positioning, shadings, shadows, decorations.pathmorphing}
\usepackage{fontawesome} 
\usepackage{pgfplots}
\pgfplotsset{compat=1.18}
\usepackage{xcolor}
\usepackage{array}
\usepackage{fontawesome, fontawesome5}
\usepackage{mathabx}
\usepackage{tcolorbox}
\tcbuselibrary{skins}
\usepackage{hyperref} 
\usepackage{pgfplots}
\usepackage[ruled,vlined,linesnumbered,commentsnumbered,titlenotnumbered]{algorithm2e}
\usepackage{media9}
\usepackage[absolute,overlay]{textpos}



% Definir colores (en el preámbulo)
\definecolor{apriori-blue}{HTML}{00A8FF}
\definecolor{apriori-green}{HTML}{4CD137}
\definecolor{apriori-orange}{HTML}{FF9500}
% Configuración base
\setbeamercolor{background canvas}{bg=black!85}
\setbeamercolor{normal text}{fg=white!90}
\setbeamercolor{structure}{fg=apriori-blue}
\setbeamercolor{alerted text}{fg=apriori-orange}
% --- Document Start ---
\begin{document}
% Para diagramas de flujo
% \node[fill=apriori-blue!20, text=apriori-blue] {\faSearch\ Escaneo};
% \draw[->, apriori-green, thick] (nodo1) -- (nodo2);
%Tablas de resultados
% \rowcolors{1}{apriori-blue!10}{apriori-green!10}
% Ecuaciones/Algoritmos
% \begin{alertblock}{Fase de Combinación}
% \textcolor{apriori-orange}{\textbf{Join Step}: $L_k \bowtie L_k$}
% \end{alertblock}
% Definir estilo del membrete
\newtcolorbox{structurebox}{
    enhanced,
    colframe=apriori-white,
    colback=white, % Fondo del cuadro
    arc=3mm, % Radio de las esquinas
    boxrule=1.5pt, % Grosor del borde
    drop fuzzy shadow=apriori-white!50, % Opcional: sombra
    before upper={\setbeamercolor{structure}{fg=black}}, % Color del texto
    fonttitle=\bfseries\large,
    left=4mm,
    right=4mm,
    top=2mm,
    bottom=2mm
}


% Definir estilo "block" para los nodos
\tikzset{
	block/.style={
		rectangle, 
		minimum width=2cm, 
		minimum height=1cm, 
		text centered, 
		draw=black, 
		fill=blue!20, 
		rounded corners
	}
}
\makeatletter
\@namedef{ver@fontawesome5.sty}{}
\makeatother

% --- Title Information ---
\title{Potenciando el Marketing con Apriori}
\subtitle{Transformando datos en estrategias efectivas}
\author{
    \begin{tabular}{c} 
    Fernanda Flores \\ 
    Kevin Arciniegas \\ 
    Giussepe Marreros
    \end{tabular}}
    
\institute{4Geeks Academy}
\date{\today}
% --- Slide 1: Title Slide ---
\begin{frame}
    \titlepage
\end{frame}

% --- Slide 3: Impactful Introduction ---
\begin{frame}{¿Sabías esto?}
    % Cuadro de texto central en la parte superior con texto más pequeño
    \begin{center}
        \begin{tcolorbox}[
            width=0.85\textwidth,
            colback=apriori-blue!30,
            colframe=apriori-blue,
            coltext=black!95,
            arc=5mm,
            boxrule=1.5pt,
            enhanced,
            drop shadow={shadow xshift=2pt, shadow yshift=-2pt}
        ]
            \normalsize
            \textbf{Un estudio de McKinsey revela que hasta el 35\% de las ventas en Amazon provienen de algoritmos de inteligencia artificial que recomiendan productos a los clientes.}
        \end{tcolorbox}
    \end{center}
    
    \vspace{0.3cm}
    
    % Imagen de Amazon centrada debajo del texto (reducida)
    \begin{center}
        \includegraphics[width=0.6\textwidth]{product-listing.jpg}
    \end{center}
    
    \vspace{0.3cm}
    
    % Estadísticas en dos columnas debajo de la imagen (reducidas)
    \begin{columns}[T]
        \begin{column}{0.5\textwidth}
            \centering
            \begin{tikzpicture}[scale=0.8]
                \node[scale=1.5, text=apriori-orange] {\Large\faChartLine};
                \node[right=0.5cm, text=white!95, font=\small] {35\% de ingresos de Amazon};
            \end{tikzpicture}
        \end{column}
        
        \begin{column}{0.5\textwidth}
            \centering
            \begin{tikzpicture}[scale=0.8]
                \node[scale=1.5, text=apriori-green] {\Large\faUsers};
                \node[right=0.5cm, text=white!95, font=\small] {Mayor satisfacción};
            \end{tikzpicture}
        \end{column}
    \end{columns}
    
    % Fuente en la parte inferior
    \vspace{0.2cm}
    \begin{center}
        \tiny\textcolor{white!80}{Fuente: blog.blendee.com/es/recomendacion-de-productos-tras-la-estela-de-amazon/}
    \end{center}
\end{frame}

% --- Slide 4: Project Purpose ---
% Agregar al principio del documento, después de los otros paquetes
\begin{frame}{Fuerza Bruta y Recursos Computacionales (I)}
    % Cuadro azul con el desafío computacional
    \begin{tcolorbox}[
        colback=apriori-blue,
        colframe=apriori-blue,
        coltext=black,
        arc=5mm,
        boxrule=0pt,
        width=0.9\textwidth,
        title={\textbf{Desafío Computacional}},
        coltitle=white,
        fonttitle=\bfseries
    ]
        \begin{itemize}
            \item \textbf{Algunas implementaciones usan fuerza bruta para evaluar combinaciones.}
            \item \textbf{Complejidad exponencial en el número de ítems.}
            \item \textbf{Método muy costoso al aumentar $n$.}
        \end{itemize}
    \end{tcolorbox}
    
    \vspace{0.7cm}
    
    % Fórmula de complejidad
    \begin{center}
        \textbf{Fórmula exacta de complejidad:}
        
        \vspace{0.4cm}
        \large
        Total de operaciones = $3^n - 2^{n+1} + 1$
        
        \vspace{0.7cm}
        \textbf{Ejemplo:} Para 10 ítems, hay más de 57,000 operaciones.
    \end{center}
\end{frame}

\begin{frame}{Fuerza Bruta y Recursos Computacionales (II)}
    \begin{center}
        \begin{tcolorbox}[
            colback=orange!30,
            colframe=orange,
            coltext=black!95,
            arc=5mm,
            boxrule=1.5pt,
            width=0.8\textwidth
        ]
            \centering
            \textbf{Apriori optimiza este proceso} reduciendo drásticamente el espacio de búsqueda mediante el principio de anti-monotonía.
        \end{tcolorbox}
    \end{center}
    
    \vspace{0.7cm}
    
    \begin{itemize}
        \item \textbf{Principio clave:} "Si un conjunto no es frecuente, ningún superconjunto lo será"
        \item \textbf{Resultado:} Reducción exponencial del espacio de búsqueda
    \end{itemize}
\end{frame}

% O alternativamente, reemplazar el emoji por texto
% Modificar estas diapositivas:

% --- Slide 5: Contextualizing the Problem ---
\section{Market Basket Analysis}
\begin{frame}{Market Basket Analysis: ¿Qué es?}
    \begin{columns}[T]
        \begin{column}{0.48\textwidth}
            \textbf{Definición:}
            \begin{tcolorbox}[
                colback=apriori-blue!30,
                colframe=apriori-blue,
                coltext=black!95,
                arc=3mm,
                boxrule=1.5pt
            ]
                Técnica usada en retail y comercio electrónico para identificar asociaciones de productos que los clientes suelen comprar juntos.
            \end{tcolorbox}
            
            \vspace{0.5cm}
            \textbf{Pregunta clave:}
            \begin{itemize}
                \item \textcolor{white!90}{¿Qué patrones de compra están escondidos en los datos?}
            \end{itemize}
        \end{column}
        
        \begin{column}{0.48\textwidth}
            \centering
            \begin{tikzpicture}[scale=0.85] % Reducido el tamaño para mejor ajuste
                % Canasta de compras estilizada con mejor contraste - movida hacia la izquierda
                \fill[apriori-orange!40, rounded corners] (-2.2,-0.3) rectangle (1.4,0.8);
                
                % Productos con iconos más visibles - ajustados a la nueva posición
                \node[text=white!95, scale=1.2] at (-1.4,0.3) {\Large\faBreadSlice};
                \node[text=white!95, scale=1.2] at (-0.4,0.3) {\Large\faCoffee};
                \node[text=white!95, scale=1.2] at (0.6,0.3) {\Large\faAppleAlt};
                
                % Flecha y producto recomendado con mejor contraste - acortada
                \draw[->, thick, apriori-green, line width=2pt] (1.6,0.3) -- (2.2,0.3);
                
                % Reemplazo del icono de leche - movido hacia la izquierda
                \node[draw=apriori-green, line width=1.3pt, fill=apriori-green!50, text=white!95, 
                      rounded corners, minimum width=1.2cm, minimum height=1cm] at (3.0,0.3) {
                    \begin{tabular}{c}
                    \Large\faGlassWhiskey \\
                    \scriptsize\textbf{LECHE}
                    \end{tabular}
                };
            \end{tikzpicture}
            
            \vspace{0.8cm}
            \begin{tcolorbox}[
                colback=apriori-orange!40,
                colframe=apriori-orange,
                coltext=black!95,
                arc=3mm,
                boxrule=1.5pt
            ]
                \centering
                \textbf{Ejemplo:} "Si un cliente compra pan y café,\\
                ¿también comprará leche?"
            \end{tcolorbox}
        \end{column}
    \end{columns}
\end{frame}

\begin{frame}{¿Qué es una Regla en Apriori?}
    \begin{columns}[T]
        \begin{column}{0.5\textwidth}
            \textbf{Definición:}\\
            Una regla de asociación es una implicación de la forma:
            \begin{center}
                \colorbox{apriori-blue!90}{\textbf{X $\rightarrow$ Y}}
            \end{center}
            donde X e Y son conjuntos de ítems (itemsets) que no se solapan.
            
            \vspace{0.3cm}
            \textbf{Interpretación:}\\
            "Si un cliente compra los productos del conjunto X, entonces es probable que también compre los productos del conjunto Y"
        \end{column}
        
        \begin{column}{0.5\textwidth}
            \begin{tcolorbox}[
                colback=apriori-orange!10,
                colframe=apriori-orange,
                title=\textbf{Ejemplo Real},
                fonttitle=\bfseries
            ]
                \centering
                \includegraphics[width=0.3\textwidth]{pan.png} \includegraphics[width=0.3\textwidth]{mante.png}
                $\Rightarrow$
                \includegraphics[width=0.3\textwidth]{merme.png}
                
                \vspace{0.2cm}
                \small
                \textbf{Soporte:} 15\% de transacciones\\
                \textbf{Confianza:} 68\% de clientes\\
                \textbf{Lift:} 3.2 (asociación fuerte)
            \end{tcolorbox}
            
            \small
            \textbf{Acción:} Colocar mermelada cerca de pan y mantequilla aumentaría ventas en un 23\%.
        \end{column}
    \end{columns}
\end{frame}

\begin{frame}{Algoritmo Apriori ¿Cómo se define?}
    \textbf{Es un método inteligente para descubrir patrones ocultos en grandes conjuntos de transacciones, como las compras de un supermercado o las relaciones de los productos en tiendas online.}
    
    \vspace{0.5cm}
    \begin{columns}[T]
        \begin{column}{0.33\textwidth}
            \begin{tcolorbox}[colback=apriori-blue!20, colframe=apriori-blue, title=\textbf{Soporte}]
                \centering
                \small
                \textbf{¿Qué tan común?}\\
                \vspace{0.2cm}
                $\frac{\text{Transacciones con }X\text{ y }Y}{\text{Total de transacciones}}$
            \end{tcolorbox}
        \end{column}
        
        \begin{column}{0.33\textwidth}
            \begin{tcolorbox}[colback=apriori-orange!20, colframe=apriori-orange, title=\textbf{Confianza}]
                \centering
                \small
                \textbf{¿Qué tan seguro?}\\
                \vspace{0.2cm}
                $\frac{\text{Soporte}(X \cup Y)}{\text{Soporte}(X)}$
            \end{tcolorbox}
        \end{column}
        
        \begin{column}{0.33\textwidth}
            \begin{tcolorbox}[colback=apriori-green!20, colframe=apriori-green, title=\textbf{Lift}]
                \centering
                \small
                \textbf{¿Qué tan fuerte?}\\
                \vspace{0.2cm}
                $\frac{\text{Confianza}(X \rightarrow Y)}{\text{Soporte}(Y)}$
            \end{tcolorbox}
        \end{column}
    \end{columns}
\end{frame}

\begin{frame}{¿Cómo Funciona el Algoritmo?}
    \textbf{Pasos Clave de Apriori:}  
    \begin{itemize}
        \item \textbf{Escaneo de transacciones:} Identifica ítems frecuentes de tamaño 1 (leche).
        \item \textbf{Generación de combinaciones:} Busca pares (leche + pan), luego tríos (leche + pan + huevos).
        \item \textbf{Filtrado inteligente:} Descarta combinaciones poco frecuentes usando el soporte mínimo
        \item \textbf{Generación de reglas:} Deriva asociaciones accionables.
    \end{itemize}
\end{frame}

\begin{frame}{Pseudocódigo del Algoritmo Apriori}
    \centering
    \begin{tcolorbox}[
        width=0.95\textwidth,
        colback=blue!10!black,
        colframe=blue!50!black,
        coltext=white,
        arc=0mm,
        boxrule=0.5mm,
        title={\textbf{Algoritmo Apriori}},
        fonttitle=\small\bfseries,
        fontsize=\footnotesize
    ]
    \textbf{Entrada:} Base de datos $D$ de Instacart, soporte mínimo $min\_sup$\\
    \textbf{Salida:} Conjunto de itemsets frecuentes $L$\\[0.1em]
    
    $L_1 \leftarrow$ \{itemsets frecuentes de 1 elemento\}\\
    $k \leftarrow 2$\\
    \textbf{Repetir:}\\
    \hspace{0.5em}$C_k \leftarrow$ generar\_candidatos($L_{k-1}$)\\
    \hspace{0.5em}\textbf{Para} transacciones en $D$:\\
    \hspace{1em}\textbf{Para} candidatos en $C_k$:\\
    \hspace{1.5em}\textbf{Si} candidato está en transacción: contar candidato\\
    \hspace{0.5em}$L_k \leftarrow$ candidatos con soporte $\geq min\_sup$\\
    \hspace{0.5em}$k \leftarrow k + 1$\\
    \textbf{Hasta que} $L_{k-1}$ sea vacío\\
    \textbf{Retornar} todos los conjuntos $L_k$
    \end{tcolorbox}
\end{frame}

\begin{frame}{Generación de Reglas: Las Matemáticas Detrás}
    \centering
    \begin{tcolorbox}[
        width=0.9\textwidth,
        colback=blue!10!black,
        colframe=blue!50!black,
        coltext=white,
        arc=0mm,
        boxrule=0.5mm,
        title={\textbf{Ecuaciones Clave}}
    ]
    \begin{align*}
    \text{Regla:} & \quad X \rightarrow Y \\[0.5em]
    \text{Soporte}(X \rightarrow Y) &= \frac{\text{Transacciones con }X\text{ y }Y}{\text{Total de transacciones}} \\[0.5em]
    \text{Confianza}(X \rightarrow Y) &= \frac{\text{Soporte}(X \cup Y)}{\text{Soporte}(X)} \\[0.5em]
    \text{Lift}(X \rightarrow Y) &= \frac{\text{Confianza}(X \rightarrow Y)}{\text{Soporte}(Y)}
    \end{align*}
    \end{tcolorbox}
    
    \vspace{0.5em}
    \small
    \textbf{Ejemplo:} Si el 5\% de transacciones contienen café y galletas, el 7\% contienen café, y el 8\% contienen galletas:
    \begin{itemize}
        \item \textbf{Confianza} (Café → Galletas) = 5\%/7\% = 71\%
        \item \textbf{Lift} = 71\%/8\% = 8.9 (¡Fuerte asociación!)
    \end{itemize}
\end{frame}



\section{Implementación}
\begin{frame}{Enfoque Metodológico (I): Caso de Estudio}
    \begin{center}
        \begin{tcolorbox}[
            colback=apriori-orange!10,
            colframe=apriori-orange,
            width=0.9\textwidth,
            boxrule=1.5pt,
            arc=5mm,
            title={\textbf{Caso de Estudio: Base de datos de Instacart}},
            fonttitle=\bfseries,
            coltitle=black!90
        ]
            \begin{itemize}
                \item \textcolor{black!100}{\textbf{Volumen:} 3 millones de órdenes}
                \item \textcolor{black!100}{\textbf{Usuarios:} Más de 200,000 clientes}
                \item \textcolor{black!100}{\textbf{Productos:} 50,000 productos únicos}
                \item \textcolor{black!100}{\textbf{Categorías:} 21 departamentos y 134 pasillos}
            \end{itemize}
        \end{tcolorbox}
    \end{center}
    
    \vspace{0.5cm}
    \textbf{Desafío:} Identificar patrones de compra y asociaciones entre productos para optimizar recomendaciones y estrategias de marketing en tiempo real.
\end{frame}

\begin{frame}{Enfoque Metodológico (II): Flujo de Trabajo}
    \begin{columns}[T]
        \begin{column}{0.45\textwidth}
            \begin{tcolorbox}[
                colback=apriori-blue!10,
                colframe=apriori-blue,
                title=\textbf{Flujo de Trabajo},
                fonttitle=\bfseries,
                coltitle=black!90
            ]
                \begin{enumerate}
                    \item \textcolor{black!100}{\textbf{Python:} Preprocesamiento de datos masivos}
                    \item \textcolor{black!90}{\textbf{R:} Aplicación del algoritmo Apriori}
                    \item \textcolor{black!90}{\textbf{Análisis:} Cálculo de métricas clave}
                    \item \textcolor{black!90}{\textbf{Visualización:} Dashboards interactivos}
                \end{enumerate}
            \end{tcolorbox}
        \end{column}
        
        \begin{column}{0.55\textwidth}
            \centering
            \begin{tikzpicture}[
                node distance=1.2cm,
                process/.style={rectangle, rounded corners, minimum width=2.5cm, minimum height=0.8cm, text centered, draw=white, fill=apriori-blue!70},
                arrow/.style={thick, ->, >=stealth, draw=white}
            ]
                \node (data) [process] {\faDatabase\ Datos Crudos};
                \node (python) [process, below of=data] {\faPython\ Limpieza (Python)};
                \node (r) [process, below of=python] {\faRProject\ Apriori (R)};
                \node (metrics) [process, below of=r] {\faChartBar\ Métricas};
                
                \draw [arrow] (data) -- (python);
                \draw [arrow] (python) -- (r);
                \draw [arrow] (r) -- (metrics);
                
                \node[right=0.5cm of metrics, text=white, align=left] {
                    \textbf{Lift:} 4.2\\
                    \textbf{Confianza:} 78\%
                };
            \end{tikzpicture}
        \end{column}
    \end{columns}
    
    \vspace{0.3cm}
    \centering
    \textbf{Ventaja:} Combinamos la eficiencia de Python para el procesamiento de datos con la precisión estadística de R para el análisis de asociaciones.
\end{frame}

\begin{frame}{Ejemplos Visuales}
    \begin{columns}[T]
        \begin{column}{0.4\textwidth}
            \textbf{Gráficas interactivas:}  
            \begin{itemize}
                \item Scatter plot que muestra lift y confianza.
                \item Ejemplo: "72\% de los clientes que compran café, también compran galletas."
            \end{itemize}
        \end{column}
        
        \begin{column}{0.6\textwidth}
            \centering
            \begin{tikzpicture}[scale=0.7]
                \begin{axis}[
                    title={Reglas de Asociación},
                    xlabel={Confianza (\%)},
                    ylabel={Lift},
                    xmin=35, xmax=95,
                    ymin=1, ymax=4.5,
                    grid=both,
                    width=7cm,
                    height=5.5cm
                ]
                
                % Puntos para las reglas
                \addplot[
                    only marks,
                    mark=*,
                    mark size=3pt,
                    color=red
                ] coordinates {
                    (72, 3.2)
                    (65, 2.8)
                    (45, 1.5)
                    (85, 4.2)
                    (55, 2.1)
                };
                
                % Etiquetas ajustadas para mejor visibilidad
                \node[anchor=west, font=\tiny, text=white] at (axis cs:72,3.2) {Café→Galletas};
                \node[anchor=west, font=\tiny, text=white] at (axis cs:65,2.8) {Pan→Leche};
                \node[anchor=south west, font=\tiny, text=white] at (axis cs:45,1.5) {Huevos→Tocino};
                \node[anchor=west, font=\tiny, text=white] at (axis cs:75,4.2) {Chips→Refresco};
                \node[anchor=west, font=\tiny, text=white] at (axis cs:55,2.1) {Pasta→Salsa};
                
                \end{axis}
            \end{tikzpicture}
        \end{column}
    \end{columns}
\end{frame}

\section{Resultados}
% Diapositiva para la visualización de Productos Recomendados
\begin{frame}{Productos Recomendados Basados en Transacción}
    \centering
    % Si es necesario ajustar márgenes de la imagen, puedes utilizar la opción trim.
    \includegraphics[width=0.8\linewidth, trim=0 0 0 0, clip]{graph_recomendados.png}
    \vspace{0.5cm}
    % Botón interactivo que vincula a la versión HTML alojada en GitHub Pages.
    \href{https://ffloresm30.github.io/Visualizaciones-apriori/graph_recomendados.html}{\beamergotobutton{Ver gráfico interactivo}}
\end{frame}
% Diapositiva para la Red Interactiva
\begin{frame}{Red Interactiva de Productos Comprados y Recomendados}
    \centering
    \includegraphics[width=0.8\linewidth, trim=0 0 0 0, clip]{graph_red.png}
    \vspace{0.5cm}
    \href{https://ffloresm30.github.io/Visualizaciones-apriori/graph_red.html}{\beamergotobutton{Ver red interactiva}}
\end{frame}

\section{Conclusión}
\begin{frame}{Nuestra Promesa: Decisiones Basadas en Datos}
    \centering
    \Huge
    \textbf{De transacciones a estrategias.}
    
    \vspace{1em}
    \Large
    
    
    \vspace{1.5em}
    \begin{tikzpicture}
        \node[fill=apriori-blue!70, text=white, rounded corners, minimum width=3cm, minimum height=1cm] at (0,0) {\textbf{Innovación}};
        \node[fill=apriori-orange!70, text=white, rounded corners, minimum width=3cm, minimum height=1cm] at (4,0) {\textbf{Precisión}};
        \node[fill=apriori-green!70, text=white, rounded corners, minimum width=3cm, minimum height=1cm] at (8,0) {\textbf{Resultados}};
    \end{tikzpicture}
    
    \vspace{1.5em}
    \large
    \textit{¡Gracias por su atención!}
\end{frame}


\end{document}
